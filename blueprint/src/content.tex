% In this file you should put the actual content of the blueprint.
% It will be used both by the web and the print version.
% It should *not* include the \begin{document}
%
% If you want to split the blueprint content into several files then
% the current file can be a simple sequence of \input. Otherwise It
% can start with a \section or \chapter for instance.

\chapter{Sum Formulas}

This chapter contains classical summation formulas that serve as a demonstration
of the Lean Blueprint workflow.

\section{Sum of Natural Numbers}

\begin{theorem}[Sum of Natural Numbers]
  \label{thm:sum_naturals}
  \lean{BluePrintDemo.sum_naturals}
  \leanok
  For all $n \in \mathbb{N}$:
  \[
    \sum_{i=0}^{n} i = \frac{n(n+1)}{2}
  \]
\end{theorem}

\begin{proof}
  By induction on $n$. The base case $n = 0$ is trivial since both sides equal $0$.
  For the inductive step, assuming the formula holds for $n$, we have:
  \[
    \sum_{i=0}^{n+1} i = \sum_{i=0}^{n} i + (n+1) = \frac{n(n+1)}{2} + (n+1) = \frac{(n+1)(n+2)}{2}
  \]
  which is the formula for $n+1$.
\end{proof}

\section{Sum of Squares}

\begin{theorem}[Sum of Squares]
  \label{thm:sum_squares}
  \lean{BluePrintDemo.sum_squares}
  \leanok
  \uses{thm:sum_naturals}
  For all $n \in \mathbb{N}$:
  \[
    \sum_{i=0}^{n} i^2 = \frac{n(n+1)(2n+1)}{6}
  \]
\end{theorem}

\begin{proof}
  By induction on $n$. The base case $n = 0$ is trivial.
  For the inductive step, assuming the formula holds for $n$:
  \[
    \sum_{i=0}^{n+1} i^2 = \sum_{i=0}^{n} i^2 + (n+1)^2 = \frac{n(n+1)(2n+1)}{6} + (n+1)^2
  \]
  Simplifying:
  \[
    = \frac{n(n+1)(2n+1) + 6(n+1)^2}{6} = \frac{(n+1)(2n^2 + n + 6n + 6)}{6} = \frac{(n+1)(n+2)(2n+3)}{6}
  \]
  which is the formula for $n+1$.
\end{proof}
